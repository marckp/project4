\documentclass[10pt,showpacs,preprintnumbers,footinbib,amsmath,amssymb,aps,prl,twocolumn,groupedaddress,superscriptaddress,showkeys]{revtex4-1}
\usepackage{graphicx}
\usepackage{dcolumn}
\usepackage{bm}
\usepackage[colorlinks=true,urlcolor=blue,citecolor=blue]{hyperref}
\usepackage{color}
\usepackage{listings}
\usepackage{amsmath}
\usepackage{subcaption}
\usepackage{hyperref}
\usepackage{fancyref}
\usepackage{verbatim}
\usepackage{pgfplots}
\pgfplotsset{compat=1.14}
\begin{document}



\title[CPP2]{Computational Physics Project 4\\
\large{Studies of phase transitions in magnetic systems}}

\author{Marc K. Pestana}
\affiliation{Institute of Theoretical Astrophysics, University of Oslo}

\begin{abstract}
The goal of my project is to study Ising model in two dimensions to simulate phase transitions. In this program, I modeled the behavior of an idealized one or two dimensional lattice of spin states, each spin with two states being up or down with corresponding values of $1$ and $-1$, as the states for our system. This is called binary system since the objects at each lattice site can only take two values. For this study, the system was treated as a canonical ensemble since atoms are neither added or removed, but energy can be added via a magnetic field or an increase in temperature. However, in this model the magnetic field is zero so that only temperature variations were studied. At any instant of time, the spin states for the entire lattice take on a particular configuration or arrange of spin states.We assumed that we had a ferromagnetic ordering, viz $J> 0$.  We used periodic boundary conditions and the Metropolis algorithm only. 
To this end, I implemented the model by developing a C++ program. I found that at a given critical temperature, my model exhibited a phase transition from a magnetic phase (a system with a finite magnetic moment) to a phase with zero magnetization.
\end{abstract}

\maketitle

\section{Introduction.}
The \href{https://en.wikipedia.org/wiki/Ising_model}{Ising model} has been extremely popular, with applications spanning from studies of phase transitions to simulations in statistics. In one and two dimensions its has analytical solutions for several expectation values and it gives a qualitatively  good understanding of several types of phase transitions.

\section{Theoretical Models}
\paragraph{Summary of Ising model}
The model I employed in my studies of phase transitions at finite temperature
for magnetic systems is the so-called Ising model. The system in this case in the canonical In its simplest form the energy is expressed as\newline
\newline
\[
E = -J\sum_{<kl>}^{N} s_ls_k - B\sum_{k}^{N} s_k\newline
\]
with $s_k = �1$, N is the total number of spins, J is a coupling constant expressing
the strength of the interaction between neighboring spins and B is an external
magnetic field interacting with the magnetic moment set up by the spins.
The symbol $< kl >$ indicates that we sum over nearest neighbors only. Notice that for $J > 0$ 0 it is energetically favorable for neighboring spins to be aligned. This is consistent the tendency for canonical systems is to "seek" the lowest energy state, it is energetically favorable for neighboring spins to be aligned, since alignment means that $s_ks_l  > 0$. This plus  $J>0 \Rightarrow E < 0$ and thus a lower energy state. 
At low enough temperatures, this feature leads to a cooperative phenomenon called spontaneous magnetization. Thus, a given magnetic moment can influence the alignment of spins that are separated from the given spin by a macroscopic distance by propagation through interactions between nearest neighbors. These long range correlations between spins are associated with a long-range order in which the lattice has a net magnetization in the absence of a magnetic field.
\paragraph{Boltzmann distribution}
In order to calculate expectation values such as the mean energy $\langle E \rangle$ or magnetization $\langle M \rangle$ in statistical physics at a given temperature, I used a probability distribution called the Boltzmann distribution\newline
\newline
$P_i(\beta) = \frac{e^{-\beta E_i}}{Z}$\newline
\newline
with $\beta = 1/kT$ being the inverse temperature, $k$ the Boltzmann constant, $E_i$ is the energy of a state $i$ while $Z$ is the partition function for the canonical ensemble defined as
\[
Z = \sum_{i=1}^{M} e^{-\beta E_i}
\]
where the sum extends over all $M$ microstates . $Z$ is the normalization constant for $P_i$ so that it expresses the probability of finding the system in a given configuration enumerated by $i$.
\paragraph{Energy and Magnetization for a specific configuration}
The energy for a specific configuration $i$ is given by
\[
E_i = -J\sum_{<kl>}^{N} s_ls_k
\]
\[
M_i = sum_{1}^{N} s_i
\]
\paragraph{Configurations}
To better understand what is meant by a configuration, consider first the case
of the one-dimensional Ising model with $B = 0$. In general, a given configuration of N spins in one dimension may look like
\[
\begin{bmatrix}
\uparrow & \uparrow  & \uparrow & \cdots & \uparrow & \downarrow & \uparrow & \cdots & \uparrow  & \uparrow \\
1 & 2 & 3 & \cdots & i - 1 & i & i + 1 & \cdots & N-1 & N\\ 
\end{bmatrix}
\]
In order to illustrate these features let us further specialize to just two spins.
With two spins, since each spin takes two values only, we have 2
2 = 4 possible
arrangements of the two spins. These four possibilities are\newline
1 =$\uparrow  \uparrow$ 2 =$\uparrow  \downarrow$ 3 =$\downarrow  \uparrow$ 4 =$\downarrow  \downarrow$ \newline
\paragraph{Boundary conditions}
What is the energy of each of these configurations?
For small systems, the way we treat the ends matters. Two cases are often
used. In the first case one can employ what is called free ends. In systems with large numbers of elements, the ends contribute a dimishingly small effect on the entire system. However, since I studied small systems, I used   
Periodic boundary conditions. This means that the
neighbor to the right of $s_N$ is assumed to take the value of $s_1$. Similarly, the
neighbor to the left of $s_1$ takes the value $s_N$ . In this case the energy for the
one-dimensional lattice 2 x 2 lattice becomes
\begin{align*}
E_i &= -J*(s_1s_2 + s_2s_1) \\
M_i &= \sum_{1}^{N} s_i \\
\intertext{For each $i$ we obtain each $E_i$ as follows:} \\
E1 &= E_{\uparrow \uparrow} = -2J \\
E2 &= E_{\uparrow \downarrow} = +2J \\
E3 &= E_{\downarrow \uparrow} = +2J \\
E4 &= E_{\downarrow \downarrow} = -2J \\
\end{align*}

For a system described by the canonical ensemble, the energy is an expectation
value since we allow energy to be exchanged with the surroundings (a heat bath
with temperature T).
I calculated the mean energy and magnetization using the probability distribution $P_i$ as
\begin{align*}
\langle E \rangle &=  \frac{1}{Z} \sum_{i=1}^{M} E_i e^{\beta E_i} \\
\langle M \rangle &=  \frac{1}{Z} \sum_{i=1}^{M} M_i e^{\beta E_i} \\
\end{align*}
\paragraph{Energy and specific heat in the canonical ensemble}
The energy is proportional to the first derivative of the potential, Helmholtz
free energy. The corresponding variance is defined as
\begin{align*}
{{\sigma}_E}^2 &= \langle E^2 \rangle - {\langle E \rangle}^2 \\
&= \frac{1}{Z} \sum_{i=1}^{M} {E_i}^2 e^{\beta E_i} - (\frac{1}{Z} \sum_{i=1}^{M} E_i e^{\beta E_i})^2
\end{align*}
If we divide the latter quantity with $kT^2$ we obtain the specific heat at constant
volume. 
\[
C_V = \frac{1}{k_B T^2}(\langle  E^2 \rangle - {\langle  E  \rangle}^2)
\]
which again can be related to the second derivative of Helmholtz free energy.
\paragraph{Magnetic moments and susceptibility in the canonical ensemble}
Using the same prescription, I computed the mean  of the absolute value of the magnetic moment
with the following formula:
\begin{align*}
\langle M \rangle &= \sum_{i}^{M} | M_i | P_i(\beta) \\
&= \frac{1}{Z} \sum_{i}^{M} | M_i | e^{-\beta E_i} \\
\intertext{and the corresponding variance} \\
{{\sigma}_M}^2 &= \langle M^2 \rangle - {\langle M \rangle}^2 \\
&= \frac{1}{Z} \sum_{i=1}^{M} {M_i}^2 e^{\beta E_i} - (\frac{1}{Z}  \sum_{i=1}^{M} | M_i | e^{\beta E_i})^2 \\
\intertext{This quantity defines also the susceptibility $\chi$ which I used to calculate $\chi$:} \\
chi &= \frac{1}{k_B T^2}(\langle  M^2 \rangle - {\langle  M  \rangle}^2) 
\end{align*}


\section{Algorithms}
The Metropolis algorithm is summarized as follows.
In a calculation of the Ising model in two dimensions, the number of configurations
is given by 2 N with N = L � L the number of spins for a lattice of length L.
Fortunately, the Metropolis algorithm considers only ratios between probabilities
and we do not need to compute the partition function at all. The algorithm goes
as follows: \newline
\begin{enumerate}
\item Establish an initial state for the configuration with energy E by creating a random configuration of the spin states for the lattice.
\item Change the initial configuration by flipping e.g., one spin only. Compute
the energy of this trial state $E_t$.
\item Calculate $\delta E = E_t - E_b$.
\item If $\delta E<0$ we accept the new configuration, meaning that the energy is
lowered and we are hopefully moving towards the energy minimum at a
given temperature. Go to step 7.
\item If $\delta E > 0$, calculate $w = e^{\beta T}$
\item Compare w with a random number $r$ selected using a uniform distribution over $[0,1]$. If $r < w$, then accept the new
configuration, else we keep the old configuration.
\item The next step is to update various expectations values.
\item The steps (2)-(7) are then repeated in order to obtain a sufficently good
representation of states.
\end{enumerate}
Each sweep through the lattice (i.e. when you have summed over
all the spins in a configuration) constitutes what is called a Monte Carlo cycle. One sweep through the lattice is called a cycle or a measurement. To normalize my measurements, I divided the
various expectation values with the total number of cycles. I choose
whether you wish to divide by the number of spins to obtain the mean energy as an intensive property of the lattice. In this setting, the energy is now
the energy per spin.

\begin{tikzpicture}
\begin{axis}[
enlargelimits=0.2,
]
\addplot+ [nodes near coords,only marks,
point meta=explicit symbolic]
table [meta=label] {
x y label
0.5 0.2 1
0.2 0.1 t2
0.7 0.6 3
0.35 0.4 Y4
0.65 0.1 5
};
\end{axis}
\end{tikzpicture}

\section{Case Studies and their Methods}

\paragraph{Case 1, project 4a): For case 1, I studied a simple $2 \times 2$ lattice, analytical expressions.}
I assumed that I had only two spins in each dimension, that is $L=2$. That is I modeled a two dimensional lattice consisting of 4 spins with a total of $4^2 = 16$ states. I found the analytical expression for the partition function, the expectations values for the energy $E$,  the mean magnetization $\langle M \rangle$ , the specific heat $C_V$ and the susceptibility $\chi$ 
as functions of  $T$ using periodic boundary conditions. I used these results  as benchmark calculations for my next steps. To perform the computation of the partition function, I first computed the energy for each configuration of the 2 x 2 lattice using periodic boundary conditions. I assumed the following scaled values for the constants: $J=1$ (coupling constant)
 and $k=1$ (for the Boltzmann's constant). 
 I also assumed that the nearest neighbors implies that $s_l$ and $s_k$ take on the values both combinations from the set $\{i,i+1\}$. Therefore, the general expression for each configuration is:\newline
 \[
E_i = -J\sum_{<kl>}^{N} s_ls_k
\]
\[
= s_1s_2 + s_2s_1 + s_2s_3 + s_3s_2 + s_3s_4 + s_4s_3 + s_4s_1 + s_1s_4
\]
This formula is computed for all 16 configurations. Each configuration is summarized in a 2 x 2 matrix of up and down arrows. Each term in the energy is determined as follows:\newline
Starting with the upper left element and going clockwise add a term in the energy equation using periodic boundary conditions.
All spins are either up or down. An up spin is given the value +1, and a down spin is given the value -1.
\paragraph{computing the Energy and Magnetization for each configuration}
The configurations are organized into permutations of like spin combinations (all up, all down, 1up and 3 down and its complement, 2 up and 2 down).
\paragraph{A spins up or all spins down}
\[
\begin{bmatrix}
\uparrow & \uparrow  \\
\uparrow & \uparrow  \\ 
\end{bmatrix}
or
\begin{bmatrix}
\downarrow & \downarrow  \\
\downarrow & \downarrow  \\ 
\end{bmatrix}
\]
$2$ x $4 \choose 4$ $= 2$\newline
\begin{align*}
E_1 &= -J \times 8 \times (1)(1)&=-8J \\
E_2 &=  -J \times 8 \times (-1)(-1)&=-8J \\
M_1 &=  4 \times 1 &= 4 \\
M_2 &=  4 \times -1 &= -4 \\
\end{align*}
\paragraph{I spin down, 3 spins up}
$4 \choose 1$ $=4$
\[
\begin{bmatrix}
\downarrow & \uparrow  \\
\uparrow & \uparrow  \\ 
\end{bmatrix}
or
\begin{bmatrix}
\uparrow & \downarrow  \\
\uparrow & \uparrow  \\ 
\end{bmatrix}
or
\begin{bmatrix}
\uparrow & \uparrow  \\
\uparrow & \downarrow  \\ 
\end{bmatrix}
or
\begin{bmatrix}
\uparrow & \uparrow  \\
\downarrow & \uparrow  \\ 
\end{bmatrix}
\]
\tiny
\begin{align*}
E_3 &= -J(2(-1)(1) + 2(1)(1) + 2(1)(1) + 2(1)(-1)) &= 0J \\
E_4 &= -J(2(1)(-1) + 2(-1)(1) + 2(1)(1) + 2(1)(-1)) &= 0J \\
E_5 &= -J(2(1)(1) + 2(1)(-1) + 2(-1)(1) + 2(1)(1)) &= 0J \\
E_6 &= -J(2(1)(1) + 2(1)(1) + 2(1)(-1) + 2(-1)(1)) &= 0J \\
M_3 &= -1 + 3*1 &= 2 \\ 
M_4 &= -1 + 3*1 &= 2 \\
M_5 &= -1 + 3*1 &= 2 \\
M_6 &= -1 + 3*1 &= 2 \\
\end{align*}
\normalsize

\paragraph{3 spin down, 1 spins up}
$4 \choose 3$ $=4$
\[
\begin{bmatrix}
\uparrow & \downarrow  \\
\downarrow & \downarrow  \\ 
\end{bmatrix}
or
\begin{bmatrix}
\downarrow & \uparrow  \\
\downarrow & \downarrow  \\ 
\end{bmatrix}
or
\begin{bmatrix}
\downarrow & \downarrow  \\
\uparrow & \downarrow  \\ 
\end{bmatrix}
or
\begin{bmatrix}
\downarrow & \downarrow  \\
\downarrow & \uparrow  \\ 
\end{bmatrix}
\]
\tiny
\begin{align*}
E_7 &= -J(2(1)(-1) + 2(-1)(-1) + 2(-1)(-1) + 2(-1)(1)) &= 0J \\
E_8 &= -J(2(1)(-1) + 2(-1)(1) + 2(1)(1) + 2(1)(-1)) &= 0J \\
E_9 &= -J(2(1)(1) + 2(1)(-1) + 2(-1)(1) + 2(1)(1)) &= 0J \\
E_{10} &= -J(2(1)(1) + 2(1)(1) + 2(1)(-1) + 2(-1)(1)) &= 0J  \\
M_3 &= +1 + 3*(-1) &= -2  \\
M_4 &= +1 + 3*(-1) &= -2  \\
M_5 &= +1 + 3*(-1) &= -2 \\
M_6 &= +1 + 3*(-1) &= -2 \\
\end{align*}
\normalsize
\newline
\paragraph{2 spins up, 2 spins down $\equiv$ 2 spins down, 2 spins up}
$4 \choose 2$ $=6$\newline
Two configurations with alternating states corresponding to maximum disorder, high temperature, and highest energy
\[
\begin{bmatrix}
\uparrow & \downarrow  \\
\downarrow & \uparrow  \\ 
\end{bmatrix}
or
\begin{bmatrix}
\downarrow & \uparrow  \\
\uparrow & \downarrow  \\ 
\end{bmatrix}
\]
\small
\begin{align*}
E_11 &= -J(2(1)(-1) + 2(-1)(1) + 2(1)(-1) + 2(-1)(1))&=8J \\
E_12 &= -J(2(-1)(1) + 2(1)(-1) + 2(-1)(1) + 2(1)(-1))&=8J\\
M_11 &= 2(1) + 2(-1) &= 0\\
M_12 &= 2(-1) + 2(1) &= 0\\
\end{align*}
\paragraph{Four configurations with paired alternating states} 
\normalsize
\[
\begin{bmatrix}
\uparrow & \uparrow  \\
\downarrow & \downarrow  \\ 
\end{bmatrix}
or
\begin{bmatrix}
\downarrow & \downarrow  \\
\uparrow & \uparrow  \\ 
\end{bmatrix}
or
\begin{bmatrix}
\uparrow & \downarrow  \\
\uparrow & \downarrow  \\ 
\end{bmatrix}
or
\begin{bmatrix}
\downarrow & \uparrow  \\
 \downarrow & \uparrow  \\ 
\end{bmatrix}
\]
\small
\begin{align*}
E_{13} &= -J(2(1)(1) + 2(1)(-1) + 2(-1)(-1) + 2(-1)(1))&=0J\\
E_{14} &= -J(2(-1)(-1) + 2(-1)(1) + 2(1)(1) + 2(1)(-1))&=0J\\
E_{15} &= -J(2(1)(-1) + 2(-1)(-1) + 2(-1)(1) + 2(1)(1))&=0J\\
E_{16} &= -J(2(-1)(1) + 2(1)(1) + 2(1)(-1) + 2(-1)(-1))&=0J\\
M_{13} &= 2(1) + 2(-1) &= 0\\
M_{14} &= 2(1) + 2(-1) &= 0\\
M_{15} &= 2(1) + 2(-1) &= 0\\
M_{16} &= 2(1) + 2(-1) &= 0\\
\end{align*}

\normalsize
\begin{table}
\caption{List of spin configurations with energy and magnetic moment}
\begin{tabular}{|l|l|l|l|}
\hline
Number of up-spins & Degeneracy & Energy & Magnetization\\
\hline
4&1&-8J&4\\
\hline
3&4&0&2\\
\hline
2&4&0&0\\
\hline
2&2&8J&0\\
\hline
1&4&0&-2\\
\hline
0&1&-8J&-4\\
\hline
\end{tabular}
\label{tab:energies}
\end{table}
\paragraph{computing the partition function}
\small
\begin{align*}
Z &= \sum_{1}^{16} e^{-\frac{E_i}{kT}}=\sum_{1}^{16} e^{-\frac{E_i}{T}} \\
\intertext{which after inserting the values for $E_i$ above becomes:} \\
Z &= 4 cosh(\frac{8}{T})+12 \\
\intertext{computing the mean energy $\langle E \rangle$} \\
\langle E \rangle &= \frac{\sum_{1}^{16} E_i e^{-E_i/T}}{Z} \\
&= \frac{-16e^{+\frac{8}{T}} + 16e^{-\frac{8}{T}}}{Z} \\
&= \frac{-32(sinh(8/T)}{Z} \\
&= \frac{-8sinh(\frac{8}{T})}{cosh(\frac{8}{T})+3} \\
\intertext{computing the mean square energy $\langle E^2 \rangle$} \\
\langle E^2 \rangle &= \frac{\sum_{1}^{16} E_i^2 e^{-E_i/T}}{Z} \\
&= \frac{128e^{+8/T} + 0 + 0 + 0 + 128e^{-8/T}}{Z} \\
&= 256(cosh(8/T))/(4cosh(8/T)+12) \\
&= 256\frac{cosh(8/T)}{(4cosh(8/T)+12)} \\
&= 64\frac{cosh(8/T)}{(cosh(8/T)+3)} \\
\intertext{computing the mean magnetization} \\
\langle |M| \rangle &= \frac{1}{Z} \sum_{1}^{16} |M_i| e^{-\frac{E_i}{T}} \\
&= 4e^{+8/T}+4(2) + 0 + 0 + 0 + 4(2) + 4e^{+8/T}/Z  \\
&= \frac{(4e^{+8/T} + 4e^{+8/T} + 16)}{Z} \\
&= \frac{(8e^{+8/T} + 16)}{Z} \\
&= \frac{8(e^{+8/T} +2)}{(4cosh(8/T)+12)} \\
&= \frac{2(e^{+8/T} +2)}{(cosh(8/T)+3)} \\
\intertext{computing the mean square magnetization} \\
\langle M^2 \rangle &= \frac{1}{Z} \sum_{1}^{16} M_i^2 e^{-\frac{E_i}{T}} \\
&= 16e^{+8/T}+4(4) + 0 + 0 + 0 + 4(4) + 16e^{+8/T}/Z\\
&= 32e^{+8/T} + 32/Z\\
&= 32e^{+8/T} + 32/(4cosh(8/T)+12)\\
&= 8(e^{+8/T} + 1)/(cosh(8/T)+3) \\
\intertext{computing the Specific Heat at constant volume} \\
C_V &= \frac{1}{k_B T^2}(\langle  E^2 \rangle - {\langle  E  \rangle}^2) \\
&= (\frac{64}{T^2})\frac{(1+3\cosh(8/T))}{(\cosh(8/T)+3)^2} \\
\intertext{computing the magnetic suseptability} \\
\chi &= \frac{1}{k_B T^2}(\langle  M^2 \rangle - {\langle  M  \rangle}^2) \\
&= \frac{1}{k_B T^2}(\frac{8(e^{+8/T} + 1)}{(cosh(8/T)+3)} - (\frac{2(e^{+8/T} +2)}{(cosh(8/T)+3)})^2)\\
\end{align*}

\paragraph{Case 2, project 4b):  For case 2, a C++ program implementing the Ising model.}
I wrote a C++ program for the Ising model which computes the mean energy 
$E$, mean magnetization 
$\vert M\vert$, the specific heat $C_V$ and the susceptibility $\chi$ 
as functions of  $T$ using periodic boundary conditions for 
$L=2$ in the $x$ and $y$ directions. 
I compared my results with the expressions from a)
for  a  temperature $T=1.0$ (in units of $kT/J$). 

I determined the number of Monte Carlo cycles needed in order to achieve a good agreement.

\paragraph{Case 3, project 4c): For case 3, determine when the most likely state reached?}
I choose now a square lattice with $L=20$ spins in the $x$ and $y$ directions. 

In the previous exercise  I did not study carefully how many Monte Carlo cycles were needed in order to reach the most likely state. Here
I performed a study of the time (here it corresponds to the number 
of Monte Carlo sweeps of the lattice) needed before my model reached an equilibrium condition 
that allowed me to start computing various expectations values. My 
first iteration was a rough and plain graphical
one, where I plotted various expectations values as functions of the number of Monte Carlo cycles.

I choose the first temperature of $T=1.0$ (in units of $kT/J$) and studied the 
mean energy and magnetization (absolute value) as functions of the number of Monte Carlo cycles. I let the number of Monte Carlo cycles (sweeps per lattice) represent time.
I used both an ordered (all spins pointing in one direction) and a random
spin orientation as starting configuration. 
My program reported the number of Monte Carlo cycles do you need before you reach an equilibrium situation?
I repeated this analysis for $T=2.4$. 
I estimated, based on these values, an equilibration time, and made
 a plot of the total number of accepted configurations 
as function of the total number of Monte Carlo cycles that showed how the number of accepted configurations behaved as function of temperature $T$?

\paragraph{Project 4d): For case 4, I analyzed the probability distribution.}
I computed the probability
$P(E)$ for the previous system with $L=20$ and the same temperatures, that is at $T=1.0$ and $T=2.4$.
I computed this probability by simply counting the number of times a 
given energy appears in my computation. I started the computation after 
the steady state situation has been reached. Then, I compared my results with the computed variance in energy
$\sigma^2_E$ and compared (below in the results section) the behavior I observed. 

\paragraph{Studies of phase transitions.}
Near $T_C$ we can characterize the behavior of many physical quantities
by a power law behavior.
As an example, for the Ising class of models, 
the mean magnetization is given by
\[
  \langle M(T) \rangle \sim \left(T-T_C\right)^{\beta},
\]
where $\beta=1/8$ is a so-called critical exponent. A similar relation
applies to the heat capacity

\[
  C_V(T) \sim \left|T_C-T\right|^{\alpha},
\]
and the susceptibility
\begin{equation}
  \chi(T) \sim \left|T_C-T\right|^{\gamma},
\end{equation}
with $\alpha = 0$ and $\gamma = 7/4$.
Another important quantity is the correlation length, which is expected
to be of the order of the lattice spacing for $T>> T_C$. Because the spins
become more and more correlated as $T$ approaches $T_C$, the correlation
length increases as we get closer to the critical temperature. The divergent
behavior of $\xi$ near $T_C$ 
is

\begin{equation}
  \xi(T) \sim \left|T_C-T\right|^{-\nu}.
  \label{eq:xi}
\end{equation}
A second-order phase transition is characterized by a
correlation length which spans the whole system.
Since we are always limited to a finite lattice, $\xi$ will
be proportional with the size of the lattice. 
Through so-called finite size scaling relations
it is possible to relate the behavior at finite lattices with the 
results for an infinitely large lattice.
The critical temperature scales then as

\begin{equation}
 T_C(L)-T_C(L=\infty) = aL^{-1/\nu},
 \label{eq:tc}
\end{equation}
with  $a$ a constant and  $\nu$ defined in Eq. (\ref{eq:xi}).
We set $T=T_C$ and obtain a mean magnetisation

\begin{equation}
  \langle {\cal M}(T) \rangle \sim \left(T-T_C\right)^{\beta}
  \rightarrow L^{-\beta/\nu},
  \label{eq:scale1}
\end{equation}
a heat capacity

\begin{equation}
  C_V(T) \sim \left|T_C-T\right|^{-\gamma} \rightarrow L^{\alpha/\nu},
  \label{eq:scale2}
\end{equation}
and susceptibility

\begin{equation}
  \chi(T) \sim \left|T_C-T\right|^{-\alpha} \rightarrow L^{\gamma/\nu}.
  \label{eq:scale3}
\end{equation}


\paragraph{Project 4e): Case 5, Numerical studies of phase transitions.}
I studied the behavior of the Ising model in two dimensions close to the critical temperature as a function of the lattice size $L\times L$. I calculated the expectation values for $\langle E\rangle$ and $\langle \vert M\vert\rangle$, the specific heat
$C_V$ and the susceptibility $\chi$ as functions of $T$ for $L=40$,
$L=60$, $L=80$ and $L=100$ for $T\in [2.0,2.3]$ with a step in
temperature $\Delta T=0.05$ or smaller.  I found it convenient to narrow the domain for $T$ to blah blah blah. 

I plotted $\langle E\rangle$,
$\langle \vert M\vert\rangle$, $C_V$ and $\chi$ as functions of $T$.  blah blah blah an indication of a phase transition?  I used the absolute value
$\langle \vert M\vert\rangle$ when you evaluate $\chi$.  For these production runs I
parallelized the code using OpenMP can be used. I use optimization flags, blah blah blah, when compiling. I performed a timing analysis of some selected runs in order to see that you optimized the speedup using the parallelized code. 



\paragraph{Project 4f): Case 6, Extracting the critical temperature.}
I used Eq. (\ref{eq:tc}) and the exact result
$\nu=1$ in order to estimate $T_C$ in the thermodynamic limit 
$L\rightarrow \infty$
using my simulations with $L=40$, $L=60$, $L=100$ and $L=140$
The exact result for the critical temperature (\href{{http://journals.aps.org/pr/abstract/10.1103/PhysRev.65.117}}{after Lars Onsager}) is
$kT_C/J=2/ln(1+\sqrt{2})\approx 2.269$ with $\nu=1$.


 \section{Results}
 \paragraph{Case 1, project 4a): For case 1, benchmark calculations for my next steps.}
 
 \paragraph{Case 2, project 4b):  For case 2, comparing the C++ code implementing the Ising mode with the analytical results for case 1.}
The number cycles to reach good agreement is blah blah blah.

\paragraph{Case 4, project 4c): For case 4,  expectations values as functions of the number of Monte Carlo cycles.}
blah blah blah
PLOT
blah blah blah
The number of Monte Carlo cycles needed before an equilibrium was reached blah blah blah equilibrium time
blah blah blah
PLOT
blah blah blah
figure X showed how the number of accepted configurations behaved as function of temperature $T$

\paragraph{Project 4d): For case 4, probability distribution.}


\subsection*{Background literature}

If you wish to read more about the Ising model and statistical physics here are three suggestions.

\begin{itemize}
 \item \href{{http://www.worldscientific.com/worldscibooks/10.1142/5660}}{M. Plischke and B. Bergersen}, \emph{Equilibrium Statistical Physics}, World Scientific, see chapters 5 and 6.

  \item \href{{http://www.cambridge.org/no/academic/subjects/physics/computational-science-and-modelling/guide-monte-carlo-simulations-statistical-physics-4th-edition?format=HB}}{D. P. Landau and K. Binder}, \emph{A Guide to Monte Carlo Simulations in Statistical Physics}, Cambridge, see chapters 2,3 and 4.

  \item \href{{https://global.oup.com/academic/product/monte-carlo-methods-in-statistical-physics-9780198517979?cc=no&lang=en&}}{M. E. J. Newman and T. Barkema}, \emph{Monte Carlo Methods in Statistical Physics}, Oxford, see chapters 3 and 4.
\end{itemize}

\noindent
\subsection*{Introduction to numerical projects}

Here follows a brief recipe and recommendation on how to write a report for each
project.

\begin{itemize}
  \item Give a short description of the nature of the problem and the eventual  numerical methods you have used.

  \item Describe the algorithm you have used and/or developed. Here you may find it convenient to use pseudocoding. In many cases you can describe the algorithm in the program itself.

  \item Include the source code of your program. Comment your program properly.

  \item If possible, try to find analytic solutions, or known limits in order to test your program when developing the code.

  \item Include your results either in figure form or in a table. Remember to        label your results. All tables and figures should have relevant captions        and labels on the axes.

  \item Try to evaluate the reliabilty and numerical stability/precision of your results. If possible, include a qualitative and/or quantitative discussion of the numerical stability, eventual loss of precision etc.

  \item Try to give an interpretation of you results in your answers to  the problems.

  \item Critique: if possible include your comments and reflections about the  exercise, whether you felt you learnt something, ideas for improvements and  other thoughts you've made when solving the exercise. We wish to keep this course at the interactive level and your comments can help us improve it.

  \item Try to establish a practice where you log your work at the  computerlab. You may find such a logbook very handy at later stages in your work, especially when you don't properly remember  what a previous test version  of your program did. Here you could also record  the time spent on solving the exercise, various algorithms you may have tested or other topics which you feel worthy of mentioning.
\end{itemize}

\noindent
\subsection*{Format for electronic delivery of report and programs}

The preferred format for the report is a PDF file. You can also use DOC or postscript formats or as an ipython notebook file.  As programming language we prefer that you choose between C/C++, Fortran2008 or Python. The following prescription should be followed when preparing the report:

\begin{itemize}
  \item Use Devilry to hand in your projects, log in  at  \href{{http://devilry.ifi.uio.no}}{\nolinkurl{http://devilry.ifi.uio.no}} with your normal UiO username and password and choose either 'fys3150' or 'fys4150'. There you can load up the files within the deadline.

  \item Upload \textbf{only} the report file!  For the source code file(s) you have developed please provide us with your link to your github domain.  The report file should include all of your discussions and a list of the codes you have developed.  Do not include library files which are available at the course homepage, unless you have made specific changes to them.

  \item In your git repository, please include a folder which contains selected results. These can be in the form of output from your code for a selected set of runs and input parametxers.

  \item In this and all later projects, you should include tests (for example unit tests) of your code(s).

  \item Comments  from us on your projects, approval or not, corrections to be made  etc can be found under your Devilry domain and are only visible to you and the teachers of the course.
\end{itemize}

\noindent
Finally, 
we encourage you to work two and two together. Optimal working groups consist of 
2-3 students. For this specific report you need to hand in an individual report.



\subsection*{How to install openmpi and/or OpenMP on your PC/laptop}

If you use your own laptop, for linux/ubuntu users, you need to install two packages (alternatively use the synaptic package manager)
\begin{verbatim}
  sudo apt-get install libopenmpi-dev
  sudo apt-get install openmpi-bin
\end{verbatim}
For OS X users, install brew (after having installed xcode and gcc, needed for the 
gfortran compiler of openmpi) and then run
\begin{verbatim}
brew install open-mpi
\end{verbatim}
When compiling from the command line, depending on your choice of programming language you need to compile and link as for example
\begin{verbatim}
mpic++ -O3 -o <executable> <programname.cpp>
\end{verbatim}
if you use c++ (alternatively mpicxx) and
\begin{verbatim}
mpif90 -O3 -o <executable> <programname.f90>
\end{verbatim}
if you use Fortran90.

When running an executable, run as
\begin{verbatim}
mpirun -n 10 ./<executable>
\end{verbatim}
where -n indicates the number of processes, 10 here.

With openmpi installed, when using Qt, add to your .pro file the instructions at \href{{http://dragly.org/2012/03/14/developing-mpi-applications-in-qt-creator/}}{Svenn-Arne Dragly's site}

You may need to tell Qt where opempi is stored.

For the machines at the computer lab, openmpi is located  at /usr/lib64/openmpi/bin
Add to your .bashrc file the following
\begin{verbatim}
export PATH=/usr/lib64/openmpi/bin:$PATH 
\end{verbatim}
For Windows users we recommend to follow the instructions at the \href{{https://www.open-mpi.org/software/ompi/v1.6/ms-windows.php}}{Open MPI site}.

If you use OpenMP, for linux users, compile and link with for example
\begin{verbatim}
c++ -O3 -fopenmp -o <executable>  <programname.cpp>
\end{verbatim}
For OS X users, you need to install clang-omp using brew, that is
\begin{verbatim}
brew install clang-omp
\end{verbatim}
and then compile and link with for example
\begin{verbatim}
clang-omp++ -O3 -fopenmp -o <executable>  <programname.cpp>
\end{verbatim}

If you program in Fortran and use \textbf{gfortran}, compile as for example
\begin{verbatim}
gfortran -O3 -fopenmp -o <executable>  <programname.f90>
\end{verbatim}
If you have access to Intel's \textbf{ifort} compiler, compile as 
\begin{verbatim}
ifort -O3 -fopenmp -o <executable>  <programname.f90>
\end{verbatim}










% ------------------- end of main content ---------------

\end{document}

